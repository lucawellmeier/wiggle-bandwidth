\documentclass[12pt]{amsart}

\usepackage[utf8]{inputenc}
\usepackage[T1]{fontenc}
\usepackage{hyperref}
\usepackage[nopatch=eqnum]{microtype}
\usepackage{graphicx}
\usepackage{cleveref}
\usepackage{mathtools}

\setlength{\textwidth}{\paperwidth}
\addtolength{\textwidth}{-2in}
\calclayout

\graphicspath{{build/figures/}}

\newcommand{\R}{\mathbb{R}}
\DeclareMathOperator{\Tr}{Tr}
\DeclareMathOperator{\Te}{Te}
\DeclareMathOperator{\argmin}{argmin}

\title{Wiggling the bandwidth}
\author{Luca Wellmeier}
\begin{document}
\begin{abstract}
    We explore the possibility of using two different bandwidths in the 
    fitting and evaluation part in kernel ridgeless regression.
    The provided experimental results indicate that \textit{wiggling}
    the bandwidth in this way can provide generalization and potentially 
    be a cheap alternative to the classical Tikhonov square-norm penality.
    Finally, we propose and test a new iterative method for optimizing the 
    bandwidth parameter by employing wiggling in a cheap local parameter 
    search.
\end{abstract}
\maketitle
\tableofcontents

%%%%%%%%%%%%%%%%%%%%%%%%%%%%%%%%%%%%%%%%%%%%%%%%%%%%%%%%%%%%%%%%%%%%%%%%%%%%%%%
%%%%%%%%%%%%%%%%%%%%%%%%%%%%%%%%%%%%%%%%%%%%%%%%%%%%%%%%%%%%%%%%%%%%%%%%%%%%%%%
\section{Introduction}
%%%%%%%%%%%%%%%%%%%%%%%%%%%%%%%%%%%%%%%%%%%%%%%%%%%%%%%%%%%%%%%%%%%%%%%%%%%%%%%
%%%%%%%%%%%%%%%%%%%%%%%%%%%%%%%%%%%%%%%%%%%%%%%%%%%%%%%%%%%%%%%%%%%%%%%%%%%%%%%

% kernel ridgeless regression
Let $\mathbf X \in \R^{n \times d}$ be a matrix of $n$ training examples
$X_1, \dots, X_n \in \R^d$ in $d$-dimensional Euclidean space as rows and 
let $Y$ be the column vector of responses $y_1, \dots, y_n \in \R$.
The main actor of this paper is the kernel ridgeless regression estimator
that predicts the response on an unseen data point $X \in \R^d$ as
\begin{equation} \label{eq:ridgeless}
    \hat f(X) = \sum_{i=1}^n (\mathbf K^\dagger Y)_i K(X, X_i).
\end{equation}
Here, $\mathbf K$ denotes the kernel matrix $\mathbf K_{ij} = k(X_i, X_j)$
is the \emph{kernel matrix} obtained by evaluating a kernel function 
$k \colon \R^d \times \R^d \to \R$ on the training examples and 
$\mathbf K^\dagger$ denotes its \emph{pseudo-inverse}.
We briefly recall the motivations behind these notions.
First, the \emph{representer theorem} asserts that any minimizer to the 
least squares problem
\[ \min_{f \in \mathcal H_k} \frac 1n \sum_{i=1}^n (y_i - f(X_i))^2 \]
in the \emph{reproducing kernel Hilbert space} $\mathcal H_k$ associated
to $k$, is given by a linear combination of the form
$\sum_{i=1}^n c_i k(X,X_i)$.
This allows to apply the usual linear least squares method to a - at the 
first glance - non-linear approximation problem in a possibly 
infinite-dimensional hypothesis space by pushing the data points into 
the \emph{feature space} spanned by $K(X, \cdot) \in \mathcal H_k$ 
for $X \in \R^d$.
The kernel matrix in this setting corresponds to the empirical covariance
matrix $\mathbf X^T \mathbf X$ and classical theory suggests that, even if 
it does not possess full-rank, there will always exist a minimizer to 
the least squares problem above.
The pseudo-inverse then corresponds to the unique least-squares solution
with minimal $\mathcal H_k$-norm.
This reduction of an infinite-dimensional optimization problem to a
finite-dimensional one using kernels is generally known as the 
\emph{kernel trick}.

% new regimes of statistical learning
Recently, there was a new wave of interest into kernels sparked by the 
fundamental question of why highly over-parameterized deep neural nets
perform as well as they do given that the usual established model, 
the \emph{bias-variance trade-off}, does not capture this regime.
The idea is that in order to find a good estimator one should choose 
the complexity of the hypothesis space to be high enough to avoid 
\emph{underfitting} but simultaneously low enough to avoid 
\emph{overfitting}.
For instance, if we roughly identify the number of adjustable parameters
of a model with the complexity of the hypothesis space, we would expect
a U-shaped test error curve over the number of parameters with very high
risk at the extrema where $N_{\textrm{param}}$ is very small but also
$N_{\textrm{param}} \approx N_{\textrm{train}}$, where the model is able
to \emph{interpolate} the training data - it learns them by heart and 
the analogy is the same as in the human experience where learning by heart
does not give you true understanding.
However, modern, practically successful, deep neural nets tend to have a
parameter count that exceeds the training sample count by orders of 
magnitude.
This is not captured by classical theory so the question is two-fold:
is this the reason for their performance and 
what exactly happens in this \emph{interpolating regime}.
A good start into the literature would be the following papers on 
the \emph{double descent} phenomenon (\cite{doubledescent}) and on the
concept of \emph{benign overfitting} (\cite{benignoverfitting}).

% focus on kernel methods
In particular, there are many indications that kernel learning methods 
behave analgously to deep learning from many different perspectives.
Much evidence was provided in \cite{understandkernels}.
First, many kernels as for instance Laplacian and Gaussian are always 
able to interpolate the training data perfectly, providing a prime example
of interpolating estimators.
Unlike deep neural nets, kernel methods are easy to handle not only on 
paper but also practically due to the very small amount of hyper-parameters
and there are many tasks in which their performance is comparable to the 
latter or even exceed it.
In \cite{justinterpolate}, the authors provide experimental and theoretic 
evidendence that it is better to interpolate instead of the classically 
suggested norm penalties (\textit{à la} Tikhonov) under certain circumstances.
Now, these circumstances have not been fully understood but some necessary 
conditions seem to have pinned down:
high-dimensionality (\cite{laplaceconsistency,benignoverfitting}) 
with low effective dimension (\cite{benignoverfitting})
and properly chosen curvature of the kernel w.r.t. the nature of the data 
(\cite{justinterpolate}).
Moreover, there seems to be strong links (\cite{ntklaplacian}) between 
Laplacian kernels and ReLU feed-forward neural nets through another kernel 
called the \emph{neural tangent kernel} that is able to capture the 
learning dynamics of gradient descent there.

% choice of kernel and what we will do
These considerations have also sparked new interest in the other direction:
due to all the analogies between kernels and neural nets researchers have
started to scale up kernel methods to deal with big data 
(see for instance \cite{falkon}).
Interesting targets have been kernels coming from the Sobolev-Hilbert spaces
$H^s(\R^d)$ with real smoothness index: the \emph{Matérn kernels}.
This family's smoothness limit member is the Gaussian kernel, but we will 
(due to the motivations above) 
consider only the well known the Laplacian kernel
\[ k_\gamma(x,x') = \exp(-\gamma \| x - x' \|) \]
which belongs to the less smooth part of the spectrum.
The only hyperparameter to adjust is the bandwidth $\gamma$ and in this 
paper we investigate experimentally a new principle on how to both optimize 
it in large-scale contexts and use it for regularization purposes.

The full code of the experiments can be found 
on GitHub\footnote{\url{https://github.com/lucawellmeier/wiggle-bandwidth}}.

%%%%%%%%%%%%%%%%%%%%%%%%%%%%%%%%%%%%%%%%%%%%%%%%%%%%%%%%%%%%%%%%%%%%%%%%%%%%%%%
%%%%%%%%%%%%%%%%%%%%%%%%%%%%%%%%%%%%%%%%%%%%%%%%%%%%%%%%%%%%%%%%%%%%%%%%%%%%%%%
\section{Wiggling}
%%%%%%%%%%%%%%%%%%%%%%%%%%%%%%%%%%%%%%%%%%%%%%%%%%%%%%%%%%%%%%%%%%%%%%%%%%%%%%%
%%%%%%%%%%%%%%%%%%%%%%%%%%%%%%%%%%%%%%%%%%%%%%%%%%%%%%%%%%%%%%%%%%%%%%%%%%%%%%%

% notation and basics
As announced, let us fix the Laplacian kernel function
\[ k_\gamma(x,x') = \exp\left(-\frac{\| x - x' \|}{\gamma} \right) \]
with bandwidth $\gamma > 0$.
The convention here is that smaller bandwidth actually decreases the 
area under the curve $x \mapsto k_0(x) \coloneqq k(0,x)$ and vice-versa.
This whole note is based on the observation that the kernel ridgeless 
estimators use the bandwidth twice in distinct places:
once in the computation of the \emph{dual coefficients}
\[ c_i = (\mathbf K_\gamma^\dagger Y)_i \]
where the kernel matrix depends on it and then
in the evaluation part
\[ \sum_i c_i k_\gamma(x,x_i) \]
where the kernel is computed to check similarities between sample 
points $x_i$ and unseen points $x$.
Note that the dependence of the dual coefficients on the bandwidth is 
not even guaranteed to be continuous and generally hard to understand
due to the appearing pseudo-inverse.
However, the evaluation part offers a different perspective, if we 
fix the dual coefficients $c_i$ and an unseen point $x$, then
the relationship
\[ \gamma \mapsto \sum_{i=1}^n c_i k_\gamma(x,x_i) \]
is smooth: indeed, $\gamma$ appears only as a factor to the non-smooth
norm in the exponential.

% what i mean by wiggling
Thus, we propose the new \emph{bandwidth-wiggled} estimators
\[ \hat f_{\gamma_0, \gamma}(x) \coloneqq 
    \sum_{i=1}^n (K_{\gamma_0} Y)^\dagger k_\gamma(x, x_i). \]
The pseudo-inverse is computed for a given base bandwidth $\gamma_0$
but the evaluation part is performed with a different bandwidth $\gamma$.
Due to the regularity argument above, we hypothesize that 
changing $\gamma_0 \to \gamma$ in the evaluation part is well-behaved.
The next chapters are devoted to experimentally verifying this, but 
also to explore possible new methods based on this estimator.

% wiggle search
The first method that we introduce here is referred to as 
\emph{wiggle search}.
Let $\Tr$ denote the training set with $n$ samples.
Split it into two non-empty parts: $\Tr = \Tr_1 \sqcup \Tr_2$
and train the a standard Laplacian ridgeless estimator only on
$\Tr_1$ with an initial guessed bandwidth $\gamma_0$.
Then extract the dual coefficients $c_i^{\gamma_0}$ from there 
and use them to define the family predictors 
\[ \hat f_{\gamma_0,\gamma} 
    \coloneqq \sum_{i \in \Tr_1} c_i^{\gamma_0} k_\gamma(\cdot, x_i) \]
for each $\gamma$ in a finite set $\Gamma \subset \R_{> 0}$.
Then select $\gamma^*$ such that 
\[ \gamma^* = \argmin_{\gamma \in \Gamma} 
    \hat R_{\Tr_2} \left(\hat f_{\gamma_0,\gamma}\right), \]
i.e. the associated wiggled estimator minimizes the empricial mean
squared error on the held-out part of the sample.
This algorithm is more efficient in comparison to computing the 
standard estimators:
The least squares computation to determine the dual coefficients is
cubic in the sample size, so as a first improvement, we reduce this 
time effectively by splitting the sample.
Moreover, existing techniques for optimizing the bandwidth
(i.e. grid search, cross-validation, ...) test one parameter value 
per least squares computation.
Here, instead, potentially thousands of parameters are tested with 
on single least squares assuming that $\hat f_{\gamma_0,\gamma}$ 
is representative of the actual standard estimator 
$\hat f_{\gamma,\gamma}$.

% computational trick
Note that the comparison of the different bandwidths can be computed 
in a highly vectorized way:
Let $\mathbf D$ denote the distance matrix of 2-norms between $\Tr_1$ and 
$\Tr_2$.
Once it is computed, we can test a new wiggled estimator 
for \emph{any} other bandwidth $\gamma$ by running
the vectorized computation
\[ \big( \mathbf c^{\gamma_0} \exp(-(1/\gamma) \mathbf D) \big)^T \]
yielding the new predictions of the inputs in $\Tr_2$.
Thus, the big hard part of the computation, i.e. finding $\mathbf D$,
needs to be done only once for size of $\Gamma$.

% iterative search
If wiggle search performs well and compares to the standard estimator 
with the optimal wiggled bandwidth (as we will see in the experiments), 
we can start considering an \emph{itertated wiggle search}:
run a wiggle search, find the optimal bandwidth, recompute the 
least squares with that bandwidth on $\Tr_1$ and repeat.
That will make up our final experiment in the paper.

% questions
Summing up, here is the questions we aim to provide experimental 
evidence for:
\begin{enumerate}
    \item How close are wiggled estimators and standard estimators with 
        the wiggle bandwidth? 
        What is the radius of $\gamma$'s around $\gamma_0$ such that one 
        can move knowledge from one to the other? 
        How to choose the wiggle range $\Gamma$?
    \item What would be a good point to split the training set in the 
        context of wiggle search?
        When does wiggle search provide an advantage over standard 
        estimators (or if at all)?
    \item Given that we move away from the interpolating state 
        (in a smooth and controlled manner) does wiggling compare to 
        Tikhonov?
        What happens with the norm/complexitie of a wiggled estimator
        compared to the standard one?
    \item Does an iterated wiggle search find the best bandwidth?
        Is it less expensive than classical model selection techniques?
\end{enumerate}

%%%%%%%%%%%%%%%%%%%%%%%%%%%%%%%%%%%%%%%%%%%%%%%%%%%%%%%%%%%%%%%%%%%%%%%%%%%%%%%
\subsection{Experiment: How close are wiggled and standard estimators?}

\begin{figure}[htp]
    \centering
    \includegraphics[width=\textwidth]{range}
    \caption{
        Standard versus wiggled estimators. 
        The green vertical line shows the base bandwidth.
        The dots indicate the minima.
        The blue line is indepdent of the base bandwidth. }
    \label{fig:range}
\end{figure}

In this first experiment we want to find out the relationship between wiggled
estimator $\hat f_{\gamma_0,\gamma}$ and standard estimator 
$\hat f_{\gamma,\gamma}$.
We choose a reasonably hard function $[3/2, 3] \to \R$, namely
\[ f(x) = \zeta(x) \sin(20 x), \]
and pick $N_{\Tr} = 30$ uniformly sampled training points.
In a range of $\Gamma = [10,50]$ we plot the performance on 
a training set (regular grid, $N_{\Te} = 1000$) of the "standard" estimator
for each $\gamma$ and the "wiggled" estimator wiggled for the same 
$\gamma$ but previously fit on a fixed base bandwidth $\gamma_0$.
We plot this 6 times for different base bandwidth.
See \cref{fig:range} for the results.

We summarize the observations:
\begin{itemize}
    \item Generally, the behavior of wiggling $\gamma$ away from $\gamma_0$
        is smooth as expected.
    \item The first four plots show that wiggling improves over standard
        estimators even if the base bandwidth is already optimal.
    \item If the base bandwidth is chosen too small, then a iterated 
        wiggle search goes in the direction of the standard optimal bandwidth.
    \item However, plot no. 5 shows that this seems not be the case in general.
\end{itemize}

%%%%%%%%%%%%%%%%%%%%%%%%%%%%%%%%%%%%%%%%%%%%%%%%%%%%%%%%%%%%%%%%%%%%%%%%%%%%%%%
\subsection{Experiment: Where to split the training set?}

\begin{figure}[htp]
    \centering
    \includegraphics[width=\textwidth]{split}
    \caption{split}
    \label{fig:split}
\end{figure}

%%%%%%%%%%%%%%%%%%%%%%%%%%%%%%%%%%%%%%%%%%%%%%%%%%%%%%%%%%%%%%%%%%%%%%%%%%%%%%%
\subsection{Experiment: Does it compare to Tikhonov?}

\begin{figure}[htp]
    \centering
    \includegraphics[width=\textwidth]{tikhonov}
    \caption{tikhonov}
    \label{fig:tikhonov}
\end{figure}

%%%%%%%%%%%%%%%%%%%%%%%%%%%%%%%%%%%%%%%%%%%%%%%%%%%%%%%%%%%%%%%%%%%%%%%%%%%%%%%
\subsection{Experiment: Iterative wiggling as cheap bandwidth optimization}

\begin{figure}[htp]
    \centering
    \includegraphics[width=\textwidth]{iterative_algo}
    \caption{iterative algo}
    \label{fig:iterative_algo}
\end{figure}

%%%%%%%%%%%%%%%%%%%%%%%%%%%%%%%%%%%%%%%%%%%%%%%%%%%%%%%%%%%%%%%%%%%%%%%%%%%%%%%
%%%%%%%%%%%%%%%%%%%%%%%%%%%%%%%%%%%%%%%%%%%%%%%%%%%%%%%%%%%%%%%%%%%%%%%%%%%%%%%
\section{Conclusions and next steps}
%%%%%%%%%%%%%%%%%%%%%%%%%%%%%%%%%%%%%%%%%%%%%%%%%%%%%%%%%%%%%%%%%%%%%%%%%%%%%%%
%%%%%%%%%%%%%%%%%%%%%%%%%%%%%%%%%%%%%%%%%%%%%%%%%%%%%%%%%%%%%%%%%%%%%%%%%%%%%%%

Finally, I propose a list of todos and questions for further research on 
the method.
\begin{itemize}
    \item How far is the reach of wiggling being indicative about the refit
        performance? That is, if I fix an initial bandwidth $\gamma_0$, 
        what is the range of $\gamma$ such that the performance of the 
        estimator $f_{\gamma_0, \gamma}$ can predict that of the estimator
        $f_\gamma$? In small logarithmic local ranges our experiments have
        shown strong similaritity of the risk curves but how far can this be 
        streched? Here, theoretical guarantees are needed to make the method
        pratically usable.
    \item The MNIST dataset have been shown in \cite{justinterpolate} to 
        be predicted best by ridgeless kernel estimators.
        How do the curves from the Tikhonov experiments look like on 
        datasets in which Tikhonov regularization is needed for better
        generalization?
    \item The part I am most interested in would be to prove high-probability
        bounds for the wiggled estimators and potentially convergence rates
        for the iterative algorithm (as it seems to be quite fast).
        Luckily, I am working in a research group (MaLGa in Genova) that 
        have very good understanding of kernel matrices of Matérn kernels
        and have already proven many bounds for kernel ridge(less) 
        regression in the past.
    \item Most importantly: check the literature for any papers on these
        methods. I haven't found anything so far but my research was not 
        extensive.
\end{itemize}

%%%%%%%%%%%%%%%%%%%%%%%%%%%%%%%%%%%%%%%%%%%%%%%%%%%%%%%%%%%%%%%%%%%%%%%%%%%%%%%
%%%%%%%%%%%%%%%%%%%%%%%%%%%%%%%%%%%%%%%%%%%%%%%%%%%%%%%%%%%%%%%%%%%%%%%%%%%%%%%
\begin{thebibliography}{20}
%%%%%%%%%%%%%%%%%%%%%%%%%%%%%%%%%%%%%%%%%%%%%%%%%%%%%%%%%%%%%%%%%%%%%%%%%%%%%%%
%%%%%%%%%%%%%%%%%%%%%%%%%%%%%%%%%%%%%%%%%%%%%%%%%%%%%%%%%%%%%%%%%%%%%%%%%%%%%%%

\bibitem{doubledescent}
Belkin, Mikhail, et al. "Reconciling modern machine learning practice and the bias-variance trade-off." arXiv preprint arXiv:1812.11118 (2018).

\bibitem{benignoverfitting}
Bartlett, Peter L., et al. "Benign overfitting in linear regression." Proceedings of the National Academy of Sciences 117.48 (2020): 30063-30070.

\bibitem{understandkernels}
Belkin, Mikhail, Siyuan Ma, and Soumik Mandal. "To understand deep learning we need to understand kernel learning." International Conference on Machine Learning. PMLR, 2018.

\bibitem{justinterpolate}
Liang, Tengyuan, and Alexander Rakhlin. "Just interpolate: Kernel “ridgeless” regression can generalize." The Annals of Statistics 48.3 (2020): 1329-1347.

\bibitem{ntklaplacian}
Geifman, Amnon, et al. "On the similarity between the laplace and neural tangent kernels." Advances in Neural Information Processing Systems 33 (2020): 1451-1461.

\bibitem{laplaceconsistency}
Rakhlin, Alexander, and Xiyu Zhai. "Consistency of interpolation with Laplace kernels is a high-dimensional phenomenon." Conference on Learning Theory. PMLR, 2019.

\bibitem{falkon}
Rudi, Alessandro, Luigi Carratino, and Lorenzo Rosasco. "Falkon: An optimal large scale kernel method." Advances in neural information processing systems 30 (2017).

\end{thebibliography}
\end{document}