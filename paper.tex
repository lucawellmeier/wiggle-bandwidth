\documentclass[12pt]{amsart}

\usepackage[utf8]{inputenc}
\usepackage[T1]{fontenc}
\usepackage{hyperref}
\usepackage[nopatch=eqnum]{microtype}
\usepackage{graphicx}
\usepackage{cleveref}

\graphicspath{{build/figures/}}

\newcommand{\R}{\mathbb{R}}

\title{Wiggling the bandwidth in ridgeless regression 
        with scale-dependent kernels}
\author{Luca Wellmeier}
\begin{document}
\begin{abstract}
    We explore the possibility of using two different bandwidths in the 
    fitting and evaluation part in kernel ridgeless regression.
    The provided experimental results indicate that \textit{wiggling}
    the bandwidth in this way can generalize and be a cheap alternative to
    the classical Tikhonov square-norm penality.
    Finally, we propose and test a new iterative method for optimizing the 
    bandwidth parameter by employing wiggling in a local search.
\end{abstract}
\maketitle
\tableofcontents

%%%%%%%%%%%%%%%%%%%%%%%%%%%%%%%%%%%%%%%%%%%%%%%%%%%%%%%%%%%%%%%%%%%%%%%%%%%%%%%
%%%%%%%%%%%%%%%%%%%%%%%%%%%%%%%%%%%%%%%%%%%%%%%%%%%%%%%%%%%%%%%%%%%%%%%%%%%%%%%
\section{Introduction}
%%%%%%%%%%%%%%%%%%%%%%%%%%%%%%%%%%%%%%%%%%%%%%%%%%%%%%%%%%%%%%%%%%%%%%%%%%%%%%%
%%%%%%%%%%%%%%%%%%%%%%%%%%%%%%%%%%%%%%%%%%%%%%%%%%%%%%%%%%%%%%%%%%%%%%%%%%%%%%%

% kernel ridgeless regression
Let $\mathbf X \in \R^{n \times d}$ be a matrix of $n$ training examples
$X_1, \dots, X_n \in \R^d$ in $d$-dimensional Euclidean space as rows and 
let $Y$ be the column vector of responses $y_1, \dots, y_n \in \R$.
The main actor of this paper is the kernel ridgeless regression estimator
that predicts the response on an unseen data point $X \in \R^d$ as
\begin{equation} \label{eq:ridgeless}
    \hat f(X) = \sum_{i=1}^n (\mathbf K^\dagger Y)_i K(X, X_i).
\end{equation}
Here, $\mathbf K$ denotes the kernel matrix $\mathbf K_{ij} = k(X_i, X_j)$
is the \emph{kernel matrix} obtained by evaluating a kernel function 
$k \colon \R^d \times \R^d \to \R$ on the training examples and 
$\mathbf K^\dagger$ denotes its \emph{pseudo-inverse}.
We briefly recall the motivations behind these notions.
First, the \emph{representer theorem} asserts that any minimizer to the 
least squares problem
\[ \min_{f \in \mathcal H_k} \frac 1n \sum_{i=1}^n (y_i - f(X_i))^2 \]
in the \emph{reproducing kernel Hilbert space} $\mathcal H_k$ associated
to $k$, is given by a linear combination of the form
$\sum_{i=1}^n c_i k(X,X_i)$.
This allows to apply the usual linear least squares method to a - at the 
first glance - non-linear approximation problem in a possibly 
infinite-dimensional hypothesis space by pushing the data points into 
the \emph{feature space} spanned by $K(X, \cdot) \in \mathcal H_k$ 
for $X \in \R^d$.
The kernel matrix in this setting corresponds to the empirical covariance
matrix $\mathbf X^T \mathbf X$ and classical theory suggests that, even if 
it does not possess full-rank, there will always exist a minimizer to 
the least squares problem above.
The pseudo-inverse then corresponds to the unique least-squares solution
with minimal $\mathcal H_k$-norm.
This reduction of an infinite-dimensional optimization problem to a
finite-dimensional one using kernels is generally known as the 
\emph{kernel trick}.

% new regimes of statistical learning
Recently, there was a new wave of interest into kernels sparked by the 
fundamental question of why highly over-parameterized deep neural nets
perform as well as they do given that the usual established model, 
the \emph{bias-variance trade-off}, does not capture this regime.
The idea is that in order to find a good estimator one should choose 
the complexity of the hypothesis space to be high enough to avoid 
\emph{underfitting} but simultaneously low enough to avoid 
\emph{overfitting}.
For instance, if we roughly identify the number of adjustable parameters
of a model with the complexity of the hypothesis space, we would expect
a U-shaped test error curve over the number of parameters with very high
risk at the extrema where $N_{\textrm{param}}$ is very small but also
$N_{\textrm{param}} \approx N_{\textrm{train}}$, where the model is able
to \emph{interpolate} the training data - it learns them by heart and 
the analogy is the same as in the human experience where learning by heart
does not give you true understanding.
However, modern, practically successful, deep neural nets tend to have a
parameter count that exceeds the training sample count by orders of 
magnitude.
This is not captured by classical theory so the question is two-fold:
is this the reason for their performance and 
what exactly happens in this \emph{interpolating regime}.
A good start into the literature would be the following papers on 
the \emph{double descent} phenomenon (\cite{doubledescent}) and on the
concept of \emph{benign overfitting} (\cite{benignoverfitting}).

% focus on kernel methods
In particular, there are many indications that kernel learning methods 
behave analgously to deep learning from many different perspectives.
Much evidence was provided in \cite{understandkernels}.
First, many kernels as for instance Laplacian and Gaussian are always 
able to interpolate the training data perfectly, providing a prime example
of interpolating estimators.
Unlike deep neural nets, kernel methods are easy to handle not only on 
paper but also practically due to the very small amount of hyper-parameters
and there are many tasks in which their performance is comparable to the 
latter or even exceed it.
In \cite{justinterpolate}, the authors provide experimental and theoretic 
evidendence that it is better to interpolate instead of the classically 
suggested norm penalties (\textit{à la} Tikhonov) under certain circumstances.
Now, these circumstances have not been fully understood but some necessary 
conditions seem to have pinned down:
high-dimensionality (\cite{laplaceconsistency,benignoverfitting}) 
with low effective dimension (\cite{benignoverfitting})
and properly chosen smoothness of the kernel w.r.t. the nature of the data 
(\cite{justinterpolate}).
Moreover, there seems to be strong links (\cite{ntklaplacian}) between 
Laplacian kernels and ReLU feed-forward neural nets through another kernel 
called the \emph{neural tangent kernel} that is able to capture the 
learning dynamics of gradient descent there.

% choice of kernel and what we will do
These considerations have also sparked new interest in the other direction:
due to all the analogies between kernels and neural nets researchers have
started to scale up kernel methods to deal with big data.
Interesting targets have been kernels coming from the Sobolev-Hilbert spaces
$H^s(\R^d)$ with real smoothness index: the \emph{Matérn kernels}.
This family's smoothest member is the Gaussian kernel, but we will (due to 
the motivations above) consider only its least smooth member: the Laplacian
kernel
\[ k_\gamma(x,x') = \exp(-\gamma \| x - x' \|). \]
The only hyper-parameter to adjust is the bandwidth $\gamma$ and in this 
paper we investigate a new principle on how to both optimize it in 
large-scale contexts and use it for regularization purposes.

%%%%%%%%%%%%%%%%%%%%%%%%%%%%%%%%%%%%%%%%%%%%%%%%%%%%%%%%%%%%%%%%%%%%%%%%%%%%%%%
%%%%%%%%%%%%%%%%%%%%%%%%%%%%%%%%%%%%%%%%%%%%%%%%%%%%%%%%%%%%%%%%%%%%%%%%%%%%%%%
\section{Wiggling}
%%%%%%%%%%%%%%%%%%%%%%%%%%%%%%%%%%%%%%%%%%%%%%%%%%%%%%%%%%%%%%%%%%%%%%%%%%%%%%%
%%%%%%%%%%%%%%%%%%%%%%%%%%%%%%%%%%%%%%%%%%%%%%%%%%%%%%%%%%%%%%%%%%%%%%%%%%%%%%%

% what i mean by wiggling

% not interpolating anymore

% computational complexity and algorithmics

% questions

%%%%%%%%%%%%%%%%%%%%%%%%%%%%%%%%%%%%%%%%%%%%%%%%%%%%%%%%%%%%%%%%%%%%%%%%%%%%%%%
\subsection{Experiment: Does it compare to Tikhonov?}



%%%%%%%%%%%%%%%%%%%%%%%%%%%%%%%%%%%%%%%%%%%%%%%%%%%%%%%%%%%%%%%%%%%%%%%%%%%%%%%
\subsection{Experiment: Where to split the training set?}

\begin{figure}[htp]
    \centering
    \includegraphics[width=\textwidth]{split}
    \caption{split}
    \label{fig:split}
\end{figure}

%%%%%%%%%%%%%%%%%%%%%%%%%%%%%%%%%%%%%%%%%%%%%%%%%%%%%%%%%%%%%%%%%%%%%%%%%%%%%%%
\subsection{Experiment: Iterative wiggling as cheap bandwidth optimization}

\begin{figure}[htp]
    \centering
    \includegraphics[width=\textwidth]{iterative_algo}
    \caption{split}
    \label{fig:iterative_algo}
\end{figure}

%%%%%%%%%%%%%%%%%%%%%%%%%%%%%%%%%%%%%%%%%%%%%%%%%%%%%%%%%%%%%%%%%%%%%%%%%%%%%%%
%%%%%%%%%%%%%%%%%%%%%%%%%%%%%%%%%%%%%%%%%%%%%%%%%%%%%%%%%%%%%%%%%%%%%%%%%%%%%%%
\section{Conclusions and future directions}
%%%%%%%%%%%%%%%%%%%%%%%%%%%%%%%%%%%%%%%%%%%%%%%%%%%%%%%%%%%%%%%%%%%%%%%%%%%%%%%
%%%%%%%%%%%%%%%%%%%%%%%%%%%%%%%%%%%%%%%%%%%%%%%%%%%%%%%%%%%%%%%%%%%%%%%%%%%%%%%



%%%%%%%%%%%%%%%%%%%%%%%%%%%%%%%%%%%%%%%%%%%%%%%%%%%%%%%%%%%%%%%%%%%%%%%%%%%%%%%
%%%%%%%%%%%%%%%%%%%%%%%%%%%%%%%%%%%%%%%%%%%%%%%%%%%%%%%%%%%%%%%%%%%%%%%%%%%%%%%
\begin{thebibliography}{20}
%%%%%%%%%%%%%%%%%%%%%%%%%%%%%%%%%%%%%%%%%%%%%%%%%%%%%%%%%%%%%%%%%%%%%%%%%%%%%%%
%%%%%%%%%%%%%%%%%%%%%%%%%%%%%%%%%%%%%%%%%%%%%%%%%%%%%%%%%%%%%%%%%%%%%%%%%%%%%%%

\bibitem{doubledescent}
Belkin, Mikhail, et al. "Reconciling modern machine learning practice and the bias-variance trade-off." arXiv preprint arXiv:1812.11118 (2018).

\bibitem{benignoverfitting}
Bartlett, Peter L., et al. "Benign overfitting in linear regression." Proceedings of the National Academy of Sciences 117.48 (2020): 30063-30070.

\bibitem{understandkernels}
Belkin, Mikhail, Siyuan Ma, and Soumik Mandal. "To understand deep learning we need to understand kernel learning." International Conference on Machine Learning. PMLR, 2018.

\bibitem{justinterpolate}
Liang, Tengyuan, and Alexander Rakhlin. "Just interpolate: Kernel “ridgeless” regression can generalize." The Annals of Statistics 48.3 (2020): 1329-1347.

\bibitem{ntklaplacian}
Geifman, Amnon, et al. "On the similarity between the laplace and neural tangent kernels." Advances in Neural Information Processing Systems 33 (2020): 1451-1461.

\bibitem{laplaceconsistency}
Rakhlin, Alexander, and Xiyu Zhai. "Consistency of interpolation with Laplace kernels is a high-dimensional phenomenon." Conference on Learning Theory. PMLR, 2019.

\end{thebibliography}
\end{document}